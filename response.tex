\documentclass[11pt]{article}

\usepackage{amsmath,latexsym,amssymb} 
\usepackage{fullpage,ifthen}
\usepackage{color,soul,colordvi}
\usepackage{mdwlist,paralist}

%\renewcommand{\baselinestretch}{1.02}

%\parskip=\smallskipamount

\sloppy

%%%%%%%%%%%%%%%%%%%%%%%%%%%%%%%%%%%%%%%%%%%%%%%%%%%%%%%%%%%%%%%%%%%%%

\newcommand{\abs}[1]{\left| #1 \right|}
\newcommand{\ceil}[1]{\left\lceil {#1} \right\rceil}
\newcommand{\floor}[1]{\left\lfloor {#1} \right\rfloor}
\newcommand{\paren}[1]{\left( #1 \right)}
\newcommand{\set}[1]{\left\{ #1 \right\}}

\newcommand{\eps}{\varepsilon}

\newcommand{\calC}{\mathcal{C}}
\newcommand{\sigmad}{\sigma_{\$}}

\newcommand{\ans}[1]{\begin{quotation}{\noindent \sl {#1}}\end{quotation}}


\newcommand{\dror}[1]{\sethlcolor{yellow}\hl{Dror: #1}}
\newcommand{\ben}[1]{\sethlcolor{green}\hl{Ben: #1}}

%%%%%%%%%%%%%%%%%%%%%%%%%%%%%%%%%%%%%%%%%%%%%%%%%%%%%%%%%%%%%%%%%%%%%

\begin{document}

\title{\textbf{Local Search Algorithms for the \\ Maximum Carpool Matching
  Problem} \\
  {\Large Response to Reviewers: Submission ALGO-D-00287}}

\author{Gilad Kutiel \and Dror Rawitz}

\maketitle

%%%%%%%%%%%%%%%%%%%%%%%%%%%%%%%%%%%%%%%%%%%%%%%%%%%%%%%%%%%%%%%%%%%%%

We thank the referee for the time that was spent on reviewing our
paper and for their thoughtful comments.  The paper was revised
according to these comments.

%%%%%%%%%%%%%%%%%%%%%%%%%%%%%%%%%%%%%%%%%%%%%%%%%%%%%%%%%%%%%%%%%%%%%

\medskip

\section*{Response to Review}

\begin{enumerate}

\item \textbf{Comment}: There are multiple graphs used throughout the
  paper: a directed input graph, the related undirected graph
  obtained by ignoring edge orientations, and a graph $H$ that relates
  stars to one another. It would help the presentation if it were made
  completely clear at each stage which graph is being discussed.
  
\item[] \textbf{Response}: We now use calligraphic letters in the
  definition of the star graph, namely the star graph is denoted by
  $\mathcal{H} = (\mathcal{S}(M^*),\mathcal{E})$.
  
\item \textbf{Comment}: In the proof of Lemma 1, I do not see how the
  condition that $M_S \cup M_T = M(S \cup T) \cup M(S \cap T)$ is
  sufficient to finish the proof when $M(S \cup T)$ and $M(S \cap T)$
  are not disjoint. In that case the weights of elements in the
  intersection will be counted only once. The later proof seems to
  actually account for this.

\item[] \textbf{Response}: We now use $M(S \cup T) \uplus M(S \cap
  T)$.
  
\item \textbf{Comment}: In the proof of Lemma 8, what is meant by an
  edge of $M$ ”intersecting” $V(\mathcal{T})$?  Presumably you mean an
  edge of $M$ that is incident on one of the vertices of
  $V(\mathcal{T})$?

\item[] \textbf{Response}: We added a sentence about this issue in the
  beginning of the proof of Lemma~8.

\item \textbf{Comment}: It would be good to emphasize that the set of
  stars induced by $M^*$ includes also all of the single-vertex
  ``stars -- corresponding to $Z_M$.  This was not so clear to me on
  the first reading.
  
\item[] \textbf{Response}: This is now emphasized in the paragraph
  that defines $\mathcal{S}(M)$.

\item \textbf{Comment}: In the proof of Lemma~9, please point out the
  usage of Lemma~6, which bounds the degree of $H$, when you use Lemma~7.

\item[] \textbf{Response}: Done.
  
\item \textbf{Comment}: In the proof of Lemma~9, you write ”Since each
  such vertex set is connected and contains at most
  $\frac{k}{c_{\max}}$ stars, it follows that $|A(\mathcal{X})| \leq
  k$.''  However, I do not see how the vertex set being connected is
  relevant for $|A(\mathcal{X})|$. I agree that it is used to apply
  Lemma~8 later.

\item[] \textbf{Response}: Corrected.

\item \textbf{Comment}: In the tight example (Figure~7) it is stated
  that the solution can be improved by local search only if removes
  all of the edges.  This is not true; please see the last page of
  this document for an improvement that does not remove every edge.
  Probably one needs to rearrange the edges connecting leaves
  appropriately.  The example given by Hurkens and Schrijver%
%
  \footnote{``On the size of systems of sets every $t$ of which have an
    SDR, with an application to the worst-case ratio of heuristics for
    packing problems'', SIAM J. Discrete Math., 2(1), 6872.}
%
  might be useable here instead.  In any case, please give some
  details for a general construction or at least provide details of
  how the example can be generalized.

\item[] \textbf{Response}: We removed this figure from the paper.
  
\item \textbf{Comment}: Details should be given for the proofs of
  Theorems~5, 6, and~7.  Since the group variant already generalizes
  the standard setting why not just present proofs for the general
  case?

\item[] \textbf{Response}: We moved the material from Section 4 to
  sections 2.1, 2.2 and 3.2, respectively.

\end{enumerate}

%%%%%

\subsection*{Minor comments}

\begin{enumerate}

\item \textbf{Comment}: Page 3: The best known approximation for
  weighted $k$-set packing is in fact $\frac{k+1}{2}$, due to Piotr Berman.%
%
\footnote{``A $d/2$ Approximation for Maximum Weight Independent Set
  in $d$-Claw Free Graphs'' in SWAT 2000}

\item[] \textbf{Response}: A reference was added, and the last
  sentence of the paragraph was corrected accordingly.

\item \textbf{Comment}: Page 5, line 56: $v$ is missing from the
  definition of $N^{\text{in}}$ and $N^{\text{out}}$.

\item[] \textbf{Response}: Corrected.
  
\item \textbf{Comment}: Page 6, line 33. Again, $v$ is missing from
  $N^{\text{in}}$ and $N^{\text{out}}$.

\item[] \textbf{Response}: Corrected.
  
\item \textbf{Comment}: Page 7, caption of Figure 3: it seems that the
  last two edges listed in $\Gamma(2,3)$ are incorrect.

\item[] \textbf{Response}: The example was correct.  We corrected the
  definition of $\Gamma(u,v)$.

\item \textbf{Comment}: Page 8, Figure 5: are the double arrows in
  this figure intentional? If so, I do not understand what is meant by
  selecting ``all arcs of length 1.''

\item[] \textbf{Response}: The figure and caption were corrected.
  
\item \textbf{Comment}: Page 8, line 34: $N_{\overline{S}}$ is used
  here instead of the previous notation $N^{\text{in}}_S(v)$.  Also,
  it might be worthwhile to point out at the end that the vertices
  will not be charged again \emph{because the stars are disjoint in
    $M^*$}.

\item[] \textbf{Response}: Corrected.

\item \textbf{Comment}: Page 12, line 28: $\deg_M(u)$ should be
  $\deg_M(v)$.
  
\item[] \textbf{Response}: Corrected.

\item \textbf{Comment}: Page 12, line 39-41: Starting with ``Observe
  that if an edge \ldots,'' I do not follow why this condition is
  helpful or necessary for $M'$ to be a valid carpool matching.

\item[] \textbf{Response}: It helps to explain the transition from
  $\deg_M(V(\mathcal{T}))$ to $|M \cap A(\mathcal{T})|$.

\item \textbf{Comment}: Page 13, line 30: at the end of the proof it
  would be good remind that the sets $A(\mathcal{T})$ considered form
  a partition of all the edges (and so also of the edges in both $M$
  and $M^*$, and summing over them completes the proof.

\item[] \textbf{Response}: Done.

\end{enumerate}

\end{document}

