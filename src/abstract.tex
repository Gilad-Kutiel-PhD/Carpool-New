The \carpool problem is a star packing problem in directed graphs.
Formally, given a directed graph $G = (V, A)$, a capacity function $c:
V \rightarrow \N$, and a weight function $w : A \rightarrow \R $, a
feasible \emph{carpool matching} is a subset of arcs, $M \subseteq A$,
such that every $v \in V$ satisfies:
\begin{inparaenum}[(i)]
\item $\din[M](v) \cdot \dout[M](v) = 0$,
\item $\din[M](v) \leq c(v)$, and 
\item $\dout[M](v) \leq 1$.
\end{inparaenum}
A vertex $v$ for which $\dout[M](v) = 1$ is a \emph{passenger}, and a
vertex for which $\dout[M](v) = 0$ is a driver who has $\din[M](v)$
passengers.  In the \carpool problem the goal is to find a matching
$M$ of maximum total weight.

The problem arises when designing an online carpool service, such as
Zimride~\cite{zimride}, which tries to connect between passengers and
drivers based on (arbitrary) similarity function.  The problem is
known to be NP-hard, even for uniform weights and infinite capacities.
%without capacity constraints.

The \gcp, is an extension of the \carpool where each vertex represents a
(unsplittable) group of passengers.
Each group might have a different size.  
Formally, each vertex $u \in V$ has a size $s(u) \in \N$,
and the constraint $\din(v) \leq c(v)$ is replaced with
$\sum_{u:(u,v) \in M} s(u) \leq c(v)$.  

We show that the \carpool can be formulated as an unconstrained
submodular maximization problem, thus it admits a
$\frac{1}{2}$-approximation algorithm.
We show that the same formulation does not hold for the \gcp,
nevertheless, we present a $\frac{1}{2} - \varepsilon$-approximation algorithm
for the \gcp as well.
For the unweighted variant of
the problems when the maximum possible capacity, $C$, is fixed, we give
a $(\frac{C + 1}{2C} - \varepsilon)$-approximation algorithm.
We also show that the problem remains APX-hard even if $C \leq 3$.