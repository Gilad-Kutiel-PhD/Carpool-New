
We now argue that the local improvement algorithm from
Section~\ref{sec:improve} can be generalized to solve \gcp.
%
The main concern when trying to adopt the algorithm to \gcp is how to
determine if a vertex can be improved.
%
With \carpool, if weights are polynomially-bounded, it was enough to
consider in the incoming arcs to a vertex $v$ in a non-increasing
order of $\delta_M$ (See proof of Theorem~\ref{thm:improve-bounded}).
This does not work anymore, since \gcp we have sizes.  In fact, given
$v$, finding the best star with respect to $\delta_M$ is
a \textsc{Knapsack} instance where the size of the knapsack is $c(v)$.
%
If weights are polynomially-bounded, then $\delta_M(e)$ is bounded for
every arc $e \in A$, and therefore this instance of \textsc{Knapsack}
can be solved in polynomial time using dynamic programming.

\begin{theorem}
Algorithm~\ref{alg:grd} is a $\half$-approximation algorithm for \gcp,
if edge weights are integral and polynomially bounded.
\end{theorem}

Using stadard scaling and rounding we obtain the following result.

\begin{theorem}
There exists a $(\half-\eps)$-approximation algorithm for \gcp,
for every $\eps \in (0,\half)$.
\end{theorem}
