Carpooling, is the sharing of car journeys so that more than one person travels
in a car.
Knapen et al.~\cite{knapen2013estimating} describe an automatic service
to match commuting trips.
Users of the service register their personal profile and a set of periodically
recurring trips, 
and the service advises registered candidates on how to combine their commuting
trips by carpooling.
The service acts in two phases. 

In the first phase, the service estimates the probability that a person $a$
traveling in person's $b$ car will be satisfied by the trip.
This is done based on personal information and feedback from users on past
rides.
The second phase is about finding a carpool matching
that maximizes the global (total expected) satisfaction.

The second phase can be modeled in terms of graph theory.
Given a directed graph $G = (V, A)$.
Each vertex $v \in V$ corresponds to a user of the service and an arc
$(u, v)$ exists if the user corresponding to vertex $u$ is willing to
commute with the user corresponding to vertex $v$.
A capacity function $ c: V \rightarrow \N $ is defined
according to the number of passengers each user can drive if she was
selected as a driver.
A weight function $w : A \rightarrow \R $ defines the amount of
satisfaction $w(u, v)$,
that user $u$ gains when riding with user $v$.

A feasible \emph{carpool matching} (matching) is a triple 
$(P, D, M)$, where $P$ and $D$ form a partition of $V$, 
and $M$ is a subset of $A \cap (P \times D)$,
under the constraints that for every driver $d \in D$, 
$\din(d) \leq c(d)$, 
and for every passenger $p \in P$, ${\dout(p) \leq 1}$.
In the \textsc{\CARPOOL{}} problem we seek for a matching $(P, D, M)$ that maximizes the
total weight of $M$.
In other words, the \textsc{\CARPOOL{}} problem is about finding a set of 
(directed toward the center) vertex disjoint stars 
that maximizes the total weights on the arcs.
Figure~\ref{fig:carpool} is an example of the \textsc{\CARPOOL{}} problem.

\begin{figure}
\centering
\newcommand{\edge}[3]{
	\draw (#1) -- (#2) node[label above] {#3};
}
\subfloat[]{
\label{subfloat:input}
\begin{tikzpicture}[every node/.style={default node}, ->, very thick]

\node(1) at(-1,3) {2};
\node(2) at(.5,.5) {3};
\node(3) at(-1,0) {1};
\node(4) at(-1,1) {3};
\node(5) at(-2,2) {0};
\node(6) at(-2,-1) {4};
\node(7) at(-3,0) {3};
\node(8) at(.5,-1) {2};
\node(9) at(0,2) {1};
\node(10) at(2,1) {4};

\edge{1}{4}{3}
\edge{2}{4}{4}
\edge{3}{4}{5}

\edge{5}{7}{2}
\edge{6}{7}{4}

\edge{8}{10}{6}
\edge{9}{10}{2}

\edge{1}{5}{4}
\edge{4}{5}{4}

\edge{2}{3}{2}
\edge{6}{3}{2}
\edge{7}{3}{2}

\edge{8}{2}{1}
\edge{10}{2}{1}

\edge{3}{8}{3}
\edge{8}{3}{3}

\edge{1}{9}{1}
\edge{4}{9}{1}

\end{tikzpicture}}
\hfill\subfloat[]{
\label{subfloat:output}
\begin{tikzpicture}[every node/.style={default node}, ->, very thick]

\begin{scope}[every node/.style={default node, draw=blue}]
\node(5) at(-2,2) {0};
\node(6) at(-2,-1) {4};

\node(2) at(.5,.5) {3};
\node(3) at(-1,0) {1};
\node(1) at(-1,3) {2};

\node(8) at(.5,-1) {2};
\node(9) at(0,2) {1};
\end{scope}

\begin{scope}[every node/.style={default node, draw=red, dashed}]
\node(7) at(-3,0) {3};

\node(4) at(-1,1) {3};

\node(10) at(2,1) {4};
\end{scope}


\edge{1}{4}{3}
\edge{2}{4}{4}
\edge{3}{4}{5}

\edge{5}{7}{2}
\edge{6}{7}{4}

\edge{8}{10}{6}
\edge{9}{10}{2}

\end{tikzpicture}}

\caption[]{
\label{fig:carpool}
A carpool matching example: 
\subref{subfloat:input} 
a directed graph with capacities on the vertices and weights on the arcs. 
\subref{subfloat:output}
a feasible matching with total weight of 26.
$P$ is the set of blue vertices, and $D$ is the set of red, dashed vertices. 
}
\end{figure}  

Hartman et al.~\cite{hartman2013optimal} proved that the
\emph{\CARPOOL{}} problem considered in this paper is NP-hard,
and that the problem remains NP-hard even for a binary weight function when
the capacity function $c(v) \leq 2$ for every vertex in $V$.
It is also worth mentioning, that in the undirected, uncapacitated, unweighted
variant of the problem, the set of drivers in an optimal solution
form a minimum dominating set.
When the set of drivers is known in advanced, however, the problem becomes
tractable and can be solved using a reduction to a flow network problem.

Agatz et al.~\cite{agatz2012optimization} outlined the optimization challenges
that arise when developing technology to support ride-sharing and survey the
related operations research models in the academic literature.  
Hartman et al.~\cite{hartman2014theory} designed several heuristic algorithms
for the \CARPOOL{} problem and compared 
their performance on real data.
Other heuristic algorithms were developed as well~\cite{knapen2014exploiting}.
Arkin et al.~\cite{arkin2004approximations}, considered other variants of
capacitated star packing where a capacity vector is given as part of the input and 
capacities need to be assigned to vertices.  

Nguyen et al.~\cite{nguyen2008approximating} considered the \textsc{spanning star forest} problem
(the undirected, uncapacitated, unweighted variant of the problem).
They proved the following results:
\begin{enumerate*}
\item
there is a polynomial-time approximation scheme for planner graphs;
\item 
there is a polynomial-time $\frac{3}{5}$-approximation algorithm for graphs;
\item 
there is a polynomial-time $\frac{1}{2}$-approximation algorithm for weighted graphs.
\end{enumerate*}
They also showed how to apply the spanning star forest model to aligning multiple
genomic sequences over a tandem duplication region.
Chen et al.~\cite{chen2007improved} improved the approximation ratio to 0.71,
and also showed that the problem can not be approximated to within a factor of
$\frac{31}{32} + \epsilon$ for any $\epsilon > 0$ under the assumption 
that $\text{P} \neq \text{NP}$.
It is not clear, however, if any of the technique used to address the
\textsc{spanning star forest} problem can be generalized to approximate the
directed capacitated variant.

In section~\ref{sec:sub} we show that the problem can be formulated as
unconstarints submodular maximization problem, thus it has a 2-approximation
algorithm.
In section~\ref{sec:local} we consider the unweighted variant of the problem.
We show that when the maximum possible capacity, $C$, is fixed, there is a 
$(\frac{C + 1}{2C} - \varepsilon)$-approximation algorithm for the any $\varepsilon > 0$.
