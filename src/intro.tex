
As traveling costs become higher and parking becomes sparse is it only
natural to share rides or to \emph{carpool}.  Originally, carpooling
was an arrangement among a group of people by which they take turns
driving the other to and from a designated location.  However, taking
turns is not essential, instead passengers can share the cost of the
ride with the driver.  Carpooling has social advantages other than
reducing the costs: it reduces fuel consumption and road congestion
and frees parking space.  
%
While in the past carpooling was usually a fixed arrangement between
friends or neighbors, the emergence of social networks has made
carpooling more dynamic and wide scale.  These days applications like
Zimride~\cite{zimride}, BlaBlaCar~\cite{blablacar},
Moovit~\cite{moovit} and even Waze~\cite{waze} are matching passengers
to drivers.

The matching process of passengers to drivers entails more than
matching the route.  Passenger satisfaction also need to be taken into
account.
%
Given several riding options (including taking their own car),
passengers have preferences.  For example, a passenger may prefer to
ride with a co-worker or a friend.  A passenger may have an opinion on
a driver that she rode with in the past.  She may prefer a non-smoker,
someone who shares her taste in music, or someone who is recommended
by others.
%
Preferences may also be computed using past information.  Knapen et
al.~\cite{knapen2013estimating} described an automatic service to
match commuting trips.  Users of the service register their personal
profile and a set of periodically recurring trips, and the service
advises registered candidates on how to combine their commuting trips
by carpooling.  The service estimates the probability that a person
$a$ traveling in person's $b$ car will be satisfied by the trip.  This
is done based on personal information and feedback from users on past
rides.

In this paper we assume that potential passenger satisfactions are
given as input and the goal is to compute an assignment of passengers
to driver so as to maximize the global satisfaction.
%
More formally, we are given a directed graph $G = (V, A)$, where each
vertex $v \in V$ corresponds to a user of the service, and an arc $(u,
v)$ exists if the user corresponding to vertex $u$ is willing to
commute with the user corresponding to vertex $v$.  We are given a
capacity function $c: V \rightarrow \N$ which bound the number of
passengers each user can drive if she was selected as a driver.  A
non-negative weight function $w : A \rightarrow \R^*+$ is used to model
the amount of satisfaction $w(u,v)$ that user $u$ gains when riding
with user $v$.
%
If $(u,v) \in A$ implies that $(v,u) \in A$ and $w(u,v) = w(v,u)$, we
call the instance \emph{undirected}.  If $w(v,u) = 1$, for every
$(v,u) \in A$, then we call the instance
\emph{unweighted}.  If $c(v) = \infty$, for every $v$, then we
call the instance \emph{uncapacitated}.

Given a directed graph $G$ and a subset $M \subseteq A$, define
$\din[M](v) \eqdf \abs{\set{u :(u,v) \in M}}$ and
$\dout[M](v) \eqdf \abs{\set{u :(v,u) \in M}}$.
%
A feasible \emph{carpool matching} is a subset of arcs, $M \subseteq
A$, such that every $v \in V$ satisfies:%
\begin{inparaenum}[(i)]
\item $\din[M](v) \cdot \dout[M](v) = 0$,
\item $\din[M](v) \leq c(v)$, and 
\item $\dout[M](v) \leq 1$.
\end{inparaenum}
A feasible carpool matching $M$ partition of $V$ as follows:
\begin{align*}
P_M & \eqdf \set{v : \dout[M](v) = 1} \\
D_M & \eqdf \set{v : \din[M](v) \geq 1} \\
Z_M & \eqdf \set{v : \dout[M](v) = \din[M](c) = 0}
~.
\end{align*}
where $P_M$ is the set of \emph{passengers}, $D_M$ is the set of
\emph{active drivers}, and $Z_M$ is the set of \emph{solo drivers}.
%
In the \carpool problem the goal is to find a matching $M$ of maximum
total weight, namely to maximize $w(M) \eqdf \sum_{(v,u) \in M}
w(v,u)$.  In other words, the \carpool problem is about finding a set
of (directed toward the center) vertex disjoint stars that maximizes
the total weight of the arcs.  
%
Figure~\ref{fig:carpool} is an example of the \carpool problem.
%
Note that in the unweighted case the goal is to find a carpool
matching $M$ that maximizes $\abs{P_M}$.
%
Moreover, observe that if $G$ is undirected, $D_M \cup Z_M$ is
a \emph{dominating set}.  Hence, in this case, an optimal carpool
matching induces an optimal dominating set and vice versa.
Since \textsc{Minimum Dominating Set} is NP-hard, if follows
that \carpool is NP-hard even if the instance is undirected,
unweigthed, and uncapacitated.

\begin{figure}
\centering
\newcommand{\edge}[3]{
	\draw (#1) -- (#2) node[label above] {#3};
}
\subfloat[]{
\label{subfloat:input}
\begin{tikzpicture}[every node/.style={default node}, ->, very thick]

\node(1) at(-1,3) {2};
\node(2) at(.5,.5) {3};
\node(3) at(-1,0) {1};
\node(4) at(-1,1) {3};
\node(5) at(-2,2) {0};
\node(6) at(-2,-1) {4};
\node(7) at(-3,0) {3};
\node(8) at(.5,-1) {2};
\node(9) at(0,2) {1};
\node(10) at(2,1) {4};

\edge{1}{4}{3}
\edge{2}{4}{4}
\edge{3}{4}{5}

\edge{5}{7}{2}
\edge{6}{7}{4}

\edge{8}{10}{6}
\edge{9}{10}{2}

\edge{1}{5}{4}
\edge{4}{5}{4}

\edge{2}{3}{2}
\edge{6}{3}{2}
\edge{7}{3}{2}

\edge{8}{2}{1}
\edge{10}{2}{1}

\edge{3}{8}{3}
\edge{8}{3}{3}

\edge{1}{9}{1}
\edge{4}{9}{1}

\end{tikzpicture}}
\hfill\subfloat[]{
\label{subfloat:output}
\begin{tikzpicture}[every node/.style={default node}, ->, very thick]

\begin{scope}[every node/.style={default node, draw=blue}]
\node(5) at(-2,2) {0};
\node(6) at(-2,-1) {4};

\node(2) at(.5,.5) {3};
\node(3) at(-1,0) {1};
\node(1) at(-1,3) {2};

\node(8) at(.5,-1) {2};
\node(9) at(0,2) {1};
\end{scope}

\begin{scope}[every node/.style={default node, draw=red, dashed}]
\node(7) at(-3,0) {3};

\node(4) at(-1,1) {3};

\node(10) at(2,1) {4};
\end{scope}


\edge{1}{4}{3}
\edge{2}{4}{4}
\edge{3}{4}{5}

\edge{5}{7}{2}
\edge{6}{7}{4}

\edge{8}{10}{6}
\edge{9}{10}{2}

\end{tikzpicture}}

\caption[]{
\label{fig:carpool}
A \carpool example: (\subref{carpool:input}) an instance containing a
directed graph with capacities on the vertices and weights on the
arcs.  (\subref{carpool:output}) a feasible matching with total weight
of 23.  $P_M$ is the set of blue vertices, and $D_M$ is the set of
red, dashed vertices, and $Z_M$ contains only the dotted, black
vertex.  }
\end{figure}  

In this paper we also consider an extension of \carpool, called \gcp,
in which each vertex represents a group of passengers, and each group
might have a different size.  Such a group may be a family or two
friends traveling together.  Formally, each vertex $u \in V$ has a
size $s(u) \in \N$, and the constraint $\din[M](v) \leq c(v)$ is
replaced with the constraint $\sum_{u:(u,v) \in M} s(u) \leq c(v)$.
%
Notice that \textsc{Knapsack} is the special case where only arcs
directed at a single vertex have non-zero (integral) weights.

%%%%%

\subparagraph{Related work.}
%
Agatz et al.~\cite{agatz2012optimization} outlined the optimization
challenges that arise when developing technology to support
ride-sharing and survey the related operations research models in the
academic literature.
%
Hartman et al.~\cite{hartman2014theory} designed
several heuristic algorithms for the \carpool problem and compared
their performance on real data.  Other heuristic algorithms were
developed as well~\cite{knapen2014exploiting}.
%
Hartman~\cite{hartman2013optimal} proved that the \carpool problem is
NP-hard even in the case where the weight function is binary and
$c(v) \leq 2$ for every $v \in V$.  In addition, Hartman presented a
natural integer linear programs and showed that if the set of drivers
is known, then an optimal assignment of passengers to drivers can be
found in polynomial time using a reduction to \textsc{Network Flow}
(see also Kutiel~\cite{kutiel2016}.)

Nguyen et al.~\cite{nguyen2008approximating} considered
the \textsc{Spanning Star Forest} problem.  A \emph{star forest} is a
graph consisting of node-disjoint star graphs.  In
the \textsc{Spanning Star Forest} problem, we are given an undirected
graph $G$, and the goal is to find a spanning subgraph which is a star
forest that maximizes the weight of edges that are covered by the star
forest.  Notice that this problem is equivalent to \carpool on
undirected and uncapacitated instances.
%
Nguyen et al.~\cite{nguyen2008approximating} provided a PTAS for
unweighted planner graphs and a polynomial-time
$\frac{3}{5}$-approximation algorithm for unweighted graphs.  They
gave an exact optimization algorithm for weighted trees, and used it
on a minimum spanning tree of the input graph to obtain a
$\frac{1}{2}$-approximation algorithm for weighted graphs.  They also
shows that it is NP-hard to approximate problem \textsc{Spanning Star
Forest} within a ratio of $\frac{259}{260}+\eps$, for any $\eps>0$.
%
%They also showed how to apply the spanning star forest model to
%aligning multiple genomic sequences over a tandem duplication region.
%
Chen et al.~\cite{CENRRS13} improved the approximation ratios from
$\frac{3}{5}$ and $\inv{2}$ to $0.71$ and $0.64$, resp.  They also
showed that the problem can not be approximated to within a factor of
$\frac{31}{32} + \eps$, for any $\eps > 0$ assuming that
$\text{P} \neq \text{NP}$.
%
Chakrabarty and Goel~\cite{ChakrabartyGoel10} improved the lower bound
to $\frac{10}{11} + \eps$.

Athanassopoulos et al.~\cite{ACKK09} improve the ratio for the
unweighted case to $\frac{193}{240} \approx 0.804$.
%
They consider a natural family of \emph{local search} algorithms
for \textsc{Spanning Star Forest}.  Such an algorithm starts with the
solution where all node are star centers.  Then, it repeatedly tries
to turn $t \leq k$ from leaves to centers and $t+1$ centers to leaves.
A change is made if it results in a feasible solution, namely if each
leave is adjacent to at least one center.  The algorithm terminates
when such changes are no longer possible.  Athanassopoulos et
al.~\cite{ACKK09} showed that such algorithms do not provide better
than $(\inv{2} + \frac{1}{2(k+2)})$-approximate spanning star forests.

Bar-Noy et al.~\cite{bar2015improved} considered the
\textsc{Minimum $2$-Path Partition} problem.
In this problem the input is a complete graph on $3k$ vertices with
non-negative edge weights, and the goal is to partition the graph into
disjoint paths of length 2.  This problem is the special case of the
undirected carpool matching where $c(v) = 2$, for every $v \in V$.
They presented two approximation algorithms, one for the weighted case
whose is ratio is $0.5833$, and another for the unweighted case whose
ratio is $\frac{3}{4}$.

Another related problem is \textsc{$k$-Set Packing}.  In this problem
one is given a collection of weighted sets, each containing as most
$k$ elements, and the goal is to find a maximum weight subcollection
of disjoint sets.  Chandra and Halld\'orsson~\cite{chandra2001greedy}
presented a $\frac{3}{2(k+1)}$-approximation algorithm for this
problem.




Consider a subset of
nodes $U$ of size at most $k$.  Observe that each subset of nodes has
an optimal internal assignment of passenger to drivers.  Let the
weight of this assignment be the profit of $U$, denote $p(U)$.  If $k
= O(1)$, then $p(U)$ can be computed for every $U$ of size at most $k$
in polynomial time.  The outcome is a \textsc{Set Packing} instance.



Arkin et al.~\cite{arkin2004approximations} considered other variants
of capacitated star packing where a capacity vector is given as part
of the input and capacities need to be assigned to vertices.



Kutiel~\cite{kutiel2016} presented a $\frac{1}{3}$-approximation
algorithm for the problem and $\frac{1}{2}$-approximation algorithm for the
unweighted variant of the problem.




%%%%%

\subparagraph{Our contribution.}
%
In Section~\ref{sec:sub} we show that the \carpool can be formulated
as unconstrained submodular maximization problem, thus it has a
$\frac{1}{2}$-approximation algorithm due
to~\cite{BFNS15,buchbinder2016deterministic}.

In Section~\ref{sec:local} we consider the unweighted variant of the
problem.  We provide a $(\frac{\cmax+1}{2\cmax} - \eps)$-approximation
algorithm, for any $\eps>0$, where $\cmax \eqdf \max_{v \in V} c(v)$
is the maximum capacity.  We also show that \carpool is APX-hard even
if $\cmax \leq 3$.
%
In Section~\ref{sec:group} we consider \gcp.  
We first show that the unconstrained submodular maximization formulation for
the \carpool does not follow for the \gcp.
We show, however, that this problem still admits
a $(\frac{1}{2} -\varepsilon)$-approximation algorithm, and argue that the local
search algorithm generalizes to \gcp as well.
