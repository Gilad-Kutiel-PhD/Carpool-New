\newcommand{\edge}[2]{
	\draw (#1) -- (#2);
}

\subfloat[]{
\label{subfloat:input}
\begin{tikzpicture}[every node/.style={default node}, ->, very thick]

\node(1) at(0,0) {a};
\node(2) at(0,1.5) {b};
\node(3) at(1.5,0) {c};

\node(4) at(4,0) {d};
\node(5) at(4,1.5) {e};
\node(6) at(2.5,0) {f};

\node(7) at(0,-1.5) {d};
\node(8) at(0,-3) {e};
\node(9) at(1.5,-1.5) {f};

\node(10) at(4,-1.5) {g};
\node(11) at(4,-3) {h};

\begin{scope}[dashed, red]
\edge{2}{1}
\edge{3}{1}

\edge{5}{4}
\edge{6}{4}

\edge{8}{7}
\edge{9}{7}

\edge{11}{10}
\end{scope}

\begin{scope}[green]
\edge{2}{5}

\edge{7}{1}
\edge{3}{9}
\end{scope}

\end{tikzpicture}}
\hfill\subfloat[]{
\label{subfloat:output}
\begin{tikzpicture}[every node/.style={default node, red}, -, very thick, green]

\node(1) at(0,1.5) {3};
\node(2) at(4,1.5) {1};
\node(3) at(0,-3) {2};
\node(4) at(4,-3) {0};

\edge{1}{2}
\edge{1}{3}

\end{tikzpicture}}
