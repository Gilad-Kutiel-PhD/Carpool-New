
We now consider a variant of the problem when, in addition to the
capacity function $c:V \to \N$, and the weight (profit) function
$w:A \to \R$, we also have a size function $s:V \to \N$, and the
constraint $\din[M] \leq c(v)$ is replaced with the constraint
$\sum_{(u,v)}s(u) \leq c(v)$.  The motivation for this variant of
the problem is when we want to allow a group of people to travel
together in the same car, thus $s$ is the size of the group.  We start
by showing that this variant of the problem does not fit the
submodular maximization formulation anymore. We show, however, that
the problem still has a $(2 +\eps)$-approximation algorithm.  Finally,
we argue that the local search algorithm can be trivially generalized
to handle the new variant of the problem while keeping the same
approximation guarantees.

%%%%%

\subsection{Submodular Maximization}

Consider the submodular maximization formulation given in Section~\ref{sec:sub}.
Figure~\ref{fig:not submodular} shows that the function $w : 2^V \to \R$ is not
submodular anymore.

\begin{figure}
\begin{center}
\input{fig-not-submodular}
\end{center}
\caption{Consider the following instance where node capacity are
written inside nodes, arcs are labeled with their size and all weights
are equal to 1.  It is easy to see that if $f(S)$ is the optimal
solution where $S$ is the set of drivers, then $f(A) + f(B) = 2 \leq 3
= f(A \cup B) + f(A \cap B)$, thus, $f$ is not submodular.}
\label{fig:not submodular}
\end{figure}


%%%%%

\subsection{A $(\frac{1}{2} - \varepsilon)$-approximation Algorithm}

%\todo[inline]{write a theorem and a proof}

We now argue that this algorithm can be generalized to solve the group carpool
matching problem.
The main concern when trying to adopt the algorithm to the group carpool
matching is how to determine if a vertex can be improved.
Observe that Remark~\ref{rem:improve} does not hold anymore.
If the maximum possible capacity is bounded, one can test for improvement of
$v$ by considering all the subsets of edges intersecting $v$. 
When the capacity is unbounded, however, it is easy to see that efficient
way to test improvement implies a solution to the Knapsack problem.
Indeed, if this is the case, one can test for improvement by using a known FPTAS
for Knapsack, this is in the cost of additional $\varepsilon$ reduction in the
approximation ratio of the algorithm.



%%%%%

\subsection{A $(\frac{2\cmax}{\cmax+1} + \eps)$-approximation Algorithm}

%\todo[inline]{complete}

We now argue that the local sear






