
We now consider \gcp which a variant of \carpool in which
%
%in addition to the capacity function $c:V \to \N$, and the weight
%function $w:A \to \R^+$, we also have
we are given a size function $s:V \to \N$, and the constraint
$\din[M](v) \leq c(v)$ is replaced with the constraint
$\sum_{u:(u,v) \in M} s(u) \leq c(v)$.
%
%The motivation for this variant of the problem is when we want to
%allow a group of people to travel together in the same car, thus $s$
%is the size of the group.

We start by showing that this variant of the problem does not fit the
submodular maximization formulation as defined for \carpool.
%
Recall the submodular maximization formulation given in
Section~\ref{sec:sub}, namely $\barw: 2^V \to \R$, where
$\barw(S) \eqdf w(M(S))$ and $M(S)$ is the maximum weight carpool
matching that satisfies $D_{M(S)} \subseteq S \subseteq V \setminus
P_{M(S)}$.
%
Figure~\ref{fig:not submodular} contains an instance that shows that
the function $\barw$ is not submodular anymore.

\begin{figure}
\begin{center}
\begin{tikzpicture}[every node/.style={default node}, ->]

\node(1) at (-2,0){};
\node(2) at (0,0){2};
\node(3) at (2,0){};
\node(4) at (0,2){};

\draw (1) -- (2) node[label above]{1};
\draw (3) -- (2) node[label above]{1};
\draw (4) -- (2) node[label above]{2};

\draw[dotted,green] 
(1.west) to[out=90,in=90] node[label above] {A}
(2.east) to[out=-90, in=-90]
(1.west)
;

\draw[dashed,red] 
(2.west) to[out=90,in=90] node[label above] {B}
(3.east) to[out=-90, in=-90]
(2.west)
;

\end{tikzpicture}
\end{center}
\caption{An unweighted \carpool instance.  Capacities are written inside
nodes, and arcs are labeled with their size.  We have that $f(A) +
f(B) = 2 < 3 = f(A \cup B) + f(A \cap B)$.}
\label{fig:not submodular}
\end{figure}

We show, however, that \gcp has a $(\half -\eps)$-approximation
algorithm by extending the local improvement algorithm from
Section~\ref{sec:improve}.
%
The main concern when trying to adopt the algorithm to \gcp is how to
determine if a vertex can be improved.
%
With \carpool, if weights are polynomially-bounded, it was enough to
consider the incoming arcs to a vertex $v$ in a non-increasing
order of $\delta_M$ (See proof of Theorem~\ref{thm:improve-bounded}).
This does not work anymore, since in the \gcp we have sizes.  In fact, given
$v$, finding the best star with respect to $\delta_M$ is
a \textsc{Knapsack} instance where the size of the knapsack is $c(v)$.
%
If weights are polynomially-bounded, then $\delta_M(e)$ is bounded for
every arc $e \in A$, and therefore this instance of \textsc{Knapsack}
can be solved in polynomial time using dynamic programming.

\begin{theorem}
Algorithm~\textbf{StarImprove} is a $\half$-approximation algorithm
for \gcp, if edge weights are integral and polynomially bounded.
\end{theorem}

Using standard scaling and rounding we obtain the following result.

\begin{theorem}
There exists a $(\half-\eps)$-approximation algorithm for \gcp,
for every $\eps \in (0,\half)$.
\end{theorem}


Finally, we show that a variant of Algorithm~\textbf{EdgeSwap} from
Section~\ref{sec:local} can be used to solve \gcp while keeping the
same approximation guarantees. 
%
The only difference is that when checking feasibility of a set of arcs
we do not compare the number of passengers to the capacity of a
driver, but rather compare the total size of the passengers to the
capacity.

\begin{theorem}
There exists a $(\half + \inv{2\cmax} - \eps)$-approximation algorithm
for unweighted \gcp, for every $\eps>0$.
\end{theorem}




