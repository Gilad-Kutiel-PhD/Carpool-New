We now consider a variant of the problem when, in addition to the capacity
function $c:V \to \N$, and the weight (profit) function $w:A \to \R$, we also
have a size function $s:A \to \N$, and the constraint $\din[M] \leq c(v)$
is replaced with the constraint $\sum_{(u,v)}s(u,v) \leq c(v)$.
The motivation for this variant of the problem is when we want to allow a group
of people to travel together in the same car, thus $s$ is the size of the group.

We start by showing that this variant of the problem does not
fit the submodular maximization formulation anymore.
We show, however, that the problem still has a 
$\frac{1}{2} - \varepsilon$-approximation algorithm.
Finally, we argue that the local search algorithm can be trivially generalized
to handle the new variant of the problem while keeping the same approximation
guarantees.

To see why the group carpool problem can not be treated as a
submodular maximization problem considered the instance given in
Figure~\ref{fig:not submodular}.

\begin{figure}
\caption{
\label{fig:not submodular}
Consider the following instance where node capacity are written inside nodes, 
arcs are labeled with their size and all weights are equal to 1.
It is easy to see that if $f(S)$ is the optimal solution where $S$ is the set of
drivers, then $f(A) + f(B) = 2 \leq 3 = f(A \cup B) + f(A \cap B)$,
thus, $f$ is not submodular. 
}
\begin{center}
\begin{tikzpicture}[every node/.style={default node}, ->]

\node(1) at (-2,0){};
\node(2) at (0,0){2};
\node(3) at (2,0){};
\node(4) at (0,2){};

\draw (1) -- (2) node[label above]{1};
\draw (3) -- (2) node[label above]{1};
\draw (4) -- (2) node[label above]{2};

\draw[dotted,green] 
(1.west) to[out=90,in=90] node[label above] {A}
(2.east) to[out=-90, in=-90]
(1.west)
;

\draw[dashed,red] 
(2.west) to[out=90,in=90] node[label above] {B}
(3.east) to[out=-90, in=-90]
(2.west)
;

\end{tikzpicture}
\end{center}
\end{figure}
