
In this section we present a $(\half + \inv{2c_{\max}}
- \eps)$-approximation algoithm for unweighted \carpool whose running
time is polynomial if $c_{\max} = O(1)$.

Let $k$ be an integer to be determined later.  The local search
algorithm maintains a feasible matching throughout its execution and
operates in iterative manner where in each iteration it tries to find
a better solution by replacing a subset of at most $k$ edges in the
current solution with another (larger) subset of edges not in the
solution.  The local search algorithm halts when no improvement can be
done.  The algorithm is described in Algorithm~\ref{alg:local}.

\begin{algorithm}
\SetKw{True}{true}
\SetKw{False}{false}
\KwIn{$G = (V, A)$, $c : V \rightarrow \N$, $k$}
\KwOut{$M$}
$M \leftarrow \emptyset$								\\
\Repeat{done}{
\label{line:outerloop}
 	$done \leftarrow{}$ \True							\\
 	\ForAll{$M' \subseteq M : |M'| \leq k$}{
 		\ForAll{$A' \subseteq A \setminus M : |A'| = |M'| + 1$}{
			\If{$M \setminus M' \cup A'$ is feasible}{
				$M \leftarrow{} M \setminus M' \cup A'$	\\
				$done \leftarrow{}$ \False				\\
			}
		}
 	}
}
\Return{$M$}

\caption{Local Search}
\label{alg:local}
\end{algorithm}

%\todo[inline]{fix running time paragraph}

We claim that the local search algorithm halts in polynomial time.
Observed that in every iteration the algorithm either improves the
value of the solution by one or otherwise it terminates.  Thus, after
a maximum of $n$ iterations the algorithm stops.  In every iteration
we examine all the subsets of edges of a fixed size and test for
feasibility, both these operations can be done in polynomial time.

Observe that a vertex $v \in D_M \cup Z_M$ is the center of
a \emph{directed star} whose leaves are the passengers in the set
$P_M(v) = \set{u : (u,v) \in M}$.  ($P_M(v) = \emptyset$, for $v \in
Z_M$.)
%
Given a carpool matching $M$, we define $\calS(M)$ to be the set of
stars that are induced by $M$.  Denote by $V(S)$ the set of vertices
of a star, i.e., if $v$ is the center $S$, then $V(S) = \{v\} \cup
P_M(v)$.  Also, let $A(S)$ be the arcs of $S$.  For
$\calT \subseteq \calS(M)$, define
$V(\calT) \eqdf \bigcup_{S \in \calT} V(S)$ and
$A(\calT) \eqdf \bigcup_{S \in \calT} A(S)$.

To analyze the approximation ratio achieved by the local search
algorithm consider an arbitrary but fixed optimal matching $M^*$.
Given the solution outputted by the local search algorithm, $M$, we
build the \emph{star graph} of the two solutions in which each node
represents a star from the optimal solution, namely from $\calS(M^*)$,
and an edge exists between two nodes if there is a star in $\calS(M)$
that intersects the two corresponding stars of the optimal solution.
%
Formally $H = (\calS(M^*), E)$ where
\[
E = \set{(S^*_i, S^*_j) : \exists S \in \calS(M), 
         V(S) \cap V(S^*_i) \neq \emptyset \land
         V(S) \cap V(S^*_j) \neq \emptyset}
~.
\]
Figure~\ref{fig:conflict} depicts a star graph.

\begin{figure}[t]
%
\begin{subfigure}{.4\linewidth}
\caption{\label{subfloat:graph}}
\input{fig-sub-graph}
\end{subfigure}
%
\hfill
%
\begin{subfigure}{.4\linewidth}
\caption{\label{subfloat:conflict}}
\newcommand{\edge}[2]{
	\draw (#1) -- (#2);
}

\begin{tikzpicture}[every node/.style={default node, red}, -, very thick, green]

\node(1) at(0,0) {3};
\node(2) at(4,0) {1};
\node(3) at(0,-1.5) {2};
\node(4) at(4,-1.5) {0};

\edge{1}{2}
\edge{1}{3}

\end{tikzpicture}
\end{subfigure}
%
\caption[]{
\subref{subfloat:graph} 
An optimal matching depicted by the dashed red edges,
and another solution depicted by the solid green edges.  
\subref{subfloat:conflict}
The conflict graph, the upper left vertex corresponds 
to the star that contains the vertices a,b,c.
The numbers inside the vertices corresponds to $d_i$.   
}
\label{fig:conflict}
\end{figure}  

\begin{lemma}
The maximum degree of $H$ is $\cmax(\cmax+1)$.
\end{lemma}
\begin{proof}
Each star in $\calS(M^*)$ consists of at most $\cmax+1$ vertices and
each such vertex can belong to a star in $\calS(M)$ containing
additional $\cmax$ vertices, each of which is located in a different
star in $\calS(M^*)$.
\end{proof}


Define $\deg_M(v) \eqdf \din[M](v) + \dout[M](v)$.  For a subset
$U \subseteq V$ of vertices define $\deg_M(U) \eqdf \sum_{v \in
U} \deg_M(u)$.

\begin{lemma}
\label{lemma:r}
Let $\calT \subseteq \calS(M^*)$ that induces a connected subgraph of
$H$.  If $\abs{A(\calT)} \leq k$, then
\[
\frac{\deg_M(V(\calT))}{\deg_{M^*}(V(\calT))} 
\geq \half + \inv{2\cmax} - \inv{2\cmax\abs{\calT}}
~.
\]
\end{lemma}
\begin{proof}
Consider the solution $M'$ obtained from the $M$ by removing all the
edges from $M$ that intersect $V(\calT)$ and adding all the edges from
$M^*$ that intersect $V(\calT)$.  Observe that if an edge $(u,v)$ in
$M^*$ intersects $V(\calT)$, then $\set{u,v} \in V(\calT)$ by the
definition of the graph $H$.  Hence, $M'$ is feasible carpool
matching.

Since $\calT$ induces a connected subgraph of $H$, the removal of
edges in $M$ that intersect $V(\calT)$ decreased $\abs{M}$ by at most
$\deg_M(V(\calT)) - \abs{\calT} + 1$.
%
On the other hand, the increase in size is exactly
 $\half \deg_{M^*}(V(\calT)) \leq \cmax \abs{\calT}$.
%
Since $\abs{A(\calT)} \leq k$, we know that this difference can not be
positive, or else, the local search algorithm would not have been
terminated.  Thus 
\(
\half \deg_{M^*}(V(\calT)) \leq \deg_M(V(\calT)) - \abs{\calT} + 1
\),
and so
\[
\frac{\deg_M(V(\calT))}{\deg_{M^*}(V(\calT))}
\geq \half + \frac{\abs{\calT} - 1}{\deg_{M^*}(V(\calT))}
\geq \half + \frac{\abs{\calT} - 1}{2\cmax \abs{\calT}}
=    \half + \inv{2\cmax} - \inv{2\cmax \abs{\calT}}
~,
\]
as required.
\end{proof}

\begin{lemma}
\label{lemma:dec}
Let $G = (V,E)$ be an undirected connected graph with maximum degree
$\Delta$, and let $\ell < \abs{V}/\Delta$.  Then $G$ can be decomposed
into connected components of size at least $\ell$, and at most
$\Delta \ell$.
\end{lemma}
\begin{proof}
Let $T$ be any spanning tree of $G$ and let $r$ be an arbitrary root.
Starting with $r$, repeatedly choose a child whose subtree is strictly
larger than $\frac{\Delta-1}{\Delta} \abs{V}$, until this is not
possible.  Hence, we have reached a vertex $v$ whose subtree contains
more than $\frac{\Delta-1}{\Delta} \abs{V}$ vertices, but the subtrees
of its children contain at most $\frac{\Delta-1}{\Delta} \abs{V}$
vertices.
%
If $v = r$, then $v$ has a child $u$ with at least $\inv{\Delta}
(\abs{V}-1) \geq \ell$ vertices.  Otherwise, there exists a child $u$
of $v$ whose subtree is of size at least
$\inv{\Delta} \abs{V} \geq \ell$.  Disconnect $u$ and the vertices in
its subtree from the graph.
%
In both cases $u$'s subtree contains at most
$\frac{\Delta-1}{\Delta} \abs{V}$ vertices, and thus at least
$\inv{\Delta} \abs{V} \geq \ell$ vertices remain in the graph.  Hence
we obtain two connected subgraphs of size at least $\ell$.
%
We repeat this procedure recursively on any component of size
strictly larger $\Delta\ell$.
\end{proof}

\iffalse %%%%% Removed

\begin{lemma}
%\label{lemma:dec}
Given a parameter $k$, a connected graph $G$ with a maximum degree $d$
can be decomposed into connected components of size at least $k$, and
at most $d(k-1) + 1$.
\end{lemma}

\begin{proof}
We give a constructive proof.  Start constructing a connected
component by adding adjacent vertices one by one to the component and
removing them from the graph.  If at some point, removing a vertex
disconnect the graph, add all the small components (of size less than
$k$) to the constructed component and decompose the large components
(of size at least $k$) recursively.  Note that removing a vertex can
break the graph to at most $d - 1$ additional components.
\end{proof}

\fi %%%%% End removed


\begin{lemma}
If $k \geq \cmax$, then $\abs{M} \geq (\half + \inv{2\cmax}
- \frac{\cmax(\cmax+1)}{2k}) \cdot \abs{M^*}$.
\end{lemma}
\begin{proof}
Consider a maximal (with respect to set inclusion) connected component
of $H$ induced by the vertices in $\calT \subseteq \calS(M^*)$.
%
If $\abs{M \cap A(\calT)} \leq k$, then it must be that $\abs{M \cap
A(\calT)} = \abs{M^* \cap A(\calT)}$, since otherwise $M \cap
A(\calT)$ could be improved.

It remains to consider a maximal component $\calT$ such that If
$\abs{M \cap A(\calT)} > k$.  Observe that since the number of edges
in $S \in \calS(M^*)$ is at most $\cmax$, it must be that
$\abs{V(\calT)} > \frac{k}{\cmax}$.
%
Due to Lemma~\ref{lemma:dec} (with $\ell = \frac{k}{\cmax}$) we can
partition $\calT$ into connected components each of which contains
between $\frac{k}{\cmax^2(\cmax+1)}$ and $\frac{k}{\cmax}$ vertices.
%
Since each such vertex set $\calX$ is connected and contains at most
$\frac{k}{\cmax}$ stars, it follows that $\abs{A(\calX)} \leq k$.
%
Due to Lemma~\ref{lemma:r} we have that
\[
\frac{\deg_M(V(\calX))}{\deg_{M^*}(V(\calX))} 
\geq \half + \inv{2\cmax} - \frac{\cmax(\cmax+1)}{2k}
~.
\]
Since $\deg_{M^*}(V(\calT)) = \sum_{\calX} \deg_{M^*}(V(\calX))$ and
$\deg_M(V(\calT)) = \sum_{\calX} \deg_M(V(\calX))$, it follows that
\[
\frac{\deg_M(V(\calT))}{\deg_{M^*}(V(\calT))}
\geq \half + \inv{2\cmax} - \frac{\cmax(\cmax+1)}{2k}
~,
\]
and thus
\[
\abs{M \cap A(\calT)}
\geq \paren{ \half + \inv{2\cmax} -
             \frac{\cmax(\cmax+1)}{2k}} \abs{M^* \cap A(\calT)}
~,
\]
and the lemma follows.
\end{proof}

By setting $k = \ceil{\cmax(\cmax+1)/2\eps}$, we get the following
result.

\begin{corollary}
\label{cor:local}
There exists a $(\half + \inv{2\cmax} - \eps)$-approximation algorithm
for \carpool, for every $\eps>0$.
\end{corollary}

We now show that Corollary~\ref{cor:local} is tight.  Consider the
example in Figure~\ref{fig:local search tight}, the example is for the
special case when $k = 11$ and $\cmax = 2$ but this example can be
generalized in a straightforward manner.  One can verify that as the
graph in the example growth, the ratio between the optimal solution
and the local search solution approaches $\frac{3}{4}$.

\begin{figure}
\begin{center}
\input{fig-local-tight}
\caption{A tight example for Corollary~\ref{cor:local}. 
The optimal solution is given by the red solid arcs while the local
search solution is given by the green, dashed arcs.  The solution can
be improved by the local search algorithm only if it removes all the
edges.}
\label{fig:local search tight}
\end{center}
\end{figure}

