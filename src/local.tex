
In this section we present a $(\frac{2c_{\max}}{c_{\max} + 1}
+ \eps)$-approximation algoithm for unweighted \carpool whose running
time is polynomial if $c_{\max} = O(1)$.

Let $k$ be an integer to be determined later.  The local search
algorithm maintains a feasible matching throughout its execution and
operates in iterative manner where in each iteration it tries to find
a better solution by replacing a subset of at most $k$ edges in the
current solution with another (larger) subset of edges not in the
solution.  The local search algorithm halts when no improvement can be
done.  The algorithm is described in Algorithm~\ref{alg:local}.

\begin{algorithm}
\SetKw{True}{true}
\SetKw{False}{false}
\KwIn{$G = (V, A)$, $c : V \rightarrow \N$, $k$}
\KwOut{$M$}
$M \leftarrow \emptyset$								\\
\Repeat{done}{
\label{line:outerloop}
 	$done \leftarrow{}$ \True							\\
 	\ForAll{$M' \subseteq M : |M'| \leq k$}{
 		\ForAll{$A' \subseteq A \setminus M : |A'| = |M'| + 1$}{
			\If{$M \setminus M' \cup A'$ is feasible}{
				$M \leftarrow{} M \setminus M' \cup A'$	\\
				$done \leftarrow{}$ \False				\\
			}
		}
 	}
}
\Return{$M$}

\caption{Local Search}
\label{alg:local}
\end{algorithm}

\todo[inline]{fix running time paragraph}

We claim that the local search algorithm halts in polynomial time.
Observed that in every iteration the algorithm either improves the
value of the solution by one or otherwise it terminates.  Thus, after
a maximum of $n$ iterations the algorithm stops.  In every iteration
we examine all the subsets of edges of a fixed size and test for
feasibility, both these operations can be done in polynomial time.

Observe that a vertex $v \in D_M \cup Z_M$ is the center of
a \emph{directed star} whose leaves are the passengers in the set
$P_M(v) = \set{u : (u,v) \in M}$.  ($P_M(v) = \emptyset$, for $v \in
Z_M$.)
%
Given a carpool matching $M$, we define $\calS(M)$ to be the set of
stars that are induced by $M$.  Denote by $V(S)$ the set of vertices
of a star, i.e., if $v$ is the center $S$, then $V(S) = \{v\} \cup
P_M(v)$.  Also, for $\calT \subseteq \calS(M)$, define
$V(\calT) \eqdf \bigcup_{S \in \calT} V(S)$.  Also, let $A(S)$ be the
edges of $S$, and define $A(\calT) \eqdf \bigcup_{S \in \calT} A(S)$.

To analyze the approximation ratio achieved by the local search
algorithm consider an arbitrary but fixed optimal matching $M^*$.
Given the solution outputted by the local search algorithm, $M$, we
build the \emph{star graph} of the two solutions in which each node
represents a star from the optimal solution, namely from $\calS(M^*)$,
and an edge exists between two nodes if there is a star in $\calS(M)$
that intersects the two corresponding stars of the optimal solution.
%
Formally $H = (\calS(M^*), E)$ where
\[
E = \set{(S^*_i, S^*_j) : \exists S \in \calS(M), 
         V(S) \cap V(S^*_i) \neq \emptyset \land
         V(S) \cap V(S^*_j) \neq \emptyset}
~.
\]
Figure~\ref{fig:conflict} depicts a star graph.

\begin{figure}[t]
\centering
\subfloat[]{
\label{subfloat:graph}
\newcommand{\edge}[2]{
	\draw (#1) -- (#2);
}

\begin{tikzpicture}[every node/.style={default node}, ->, very thick]

\node(1) at(0,0) {a};
\node(2) at(0,1.5) {b};
\node(3) at(1.5,0) {c};

\node(4) at(4,0) {d};
\node(5) at(4,1.5) {e};
\node(6) at(2.5,0) {f};

\node(7) at(0,-1.5) {h};
\node(8) at(0,-3) {i};
\node(9) at(1.5,-1.5) {j};

\node(10) at(4,-1.5) {k};
\node(11) at(4,-3) {l};

\begin{scope}[dashed, red]
\edge{2}{1}
\edge{3}{1}

\edge{5}{4}
\edge{6}{4}

\edge{8}{7}
\edge{9}{7}

\edge{11}{10}
\end{scope}

\begin{scope}[green]
\edge{2}{5}

\edge{7}{1}
\edge{3}{9}
\end{scope}

\end{tikzpicture}

}
\subfloat[]{
\label{subfloat:conflict}
\newcommand{\edge}[2]{
	\draw (#1) -- (#2);
}

\begin{tikzpicture}[every node/.style={default node, red}, -, very thick, green]

\node(1) at(0,1.5) {3};
\node(2) at(4,1.5) {1};
\node(3) at(0,-3) {2};
\node(4) at(4,-3) {0};

\edge{1}{2}
\edge{1}{3}

\end{tikzpicture}
}
\caption[]{
\subref{subfloat:input} 
An optimal matching depicted by the dashed red edges,
and another solution depicted by the solid green edges.  
\subref{subfloat:output}
The conflict graph, the upper left vertex corresponds 
to the star that contains the vertices a,b,c.
The numbers inside the vertices corresponds to $d_i$.   
}
\label{fig:conflict}
\end{figure}  


\begin{lemma}
\label{lemma:r}
Let $\calT \subseteq \calS(M^*)$ that induces a connect subgraph of
$H$.  If $\abs{A(\calT)} \leq k$, then
\[
\frac{\deg_M(V(\calT))}{\deg_{M^*}(V(\calT))} 
\geq \half + \inv{2\cmax} - \inv{2\cmax\abs{\calT}}
~.
\]
\end{lemma}
\begin{proof}
Consider the solution $M'$ obtained from the $M$ by removing all the
edges from $M$ that intersect $V(\calT)$ and adding all the edges from
$M^*$ that intersect $V(\calT)$.  Observe that if an edge $(u,v)$ in
$M^*$ intersects $V(\calT)$, then $\set{u,v} \in V(\calT)$ by the
definition of the graph $H$.  Hence, $M'$ is feasible carpool
matching.

Since $\calT$ induces a connected subgraph of $H$, the removal of
edges in $M$ that intersect $V(\calT)$ decreased $\abs{M}$ by at most
$\deg_M(V(\calT)) - \abs{\calT} + 1$.
%
On the other hand, the increase in size is exactly
 $\half \deg_{M^*}(V(\calT)) \leq \cmax \abs{\calT}$.
%
Since $\abs{A(\calT)} \leq k$, we know that this difference can not be
positive, or else, the local search algorithm would not have been
terminated.  Thus 
\(
\half \deg_{M^*}(V(\calT)) \leq \deg_M(V(\calT)) - \abs{\calT} + 1
\),
and so
\[
\frac{\deg_M(V(\calT))}{\deg_{M^*}(V(\calT))}
\geq \half + \frac{\abs{\calT} - 1}{\deg_{M^*}(V(\calT))}
\geq \half + \frac{\abs{\calT} - 1}{2\cmax \abs{\calT}}
~,
\]
as required.
\end{proof}



\begin{lemma}
If $k \geq 24\cmax$, then $\abs{M} \geq (\frac{\cmax+1}{2\cmax}
- \frac{2}{k}) \cdot \abs{M^*}$.
\end{lemma}
\begin{proof}
Consider a maximal (with respect to set inclusion) connected component
of $H$ induced by the vertices in $\calT \subseteq \calS(M^*)$.
%
If $\abs{M \cap A(\calT)} \leq k$, then it must be that $\abs{M \cap
A(\calT)} = \abs{M^* \cap A(\calT)}$, since otherwise $M \cap
A(\calT)$ could be improved.

It remains to consider a maximal component $\calT$ such that If
$\abs{M \cap A(\calT)} > k$.  Observe that since the number of edges
in a $S \in \calS(M^*)$ is at most $\cmax$, it must be that
$\abs{V(\calT)} > \frac{k}{\cmax}$.
%
Given such a component $\calT$, we partition its vertex set into
several sets, each of size at least $\frac{k}{4\cmax}$.  We do this as
follows.  First construct any spanning tree of $\calT$ (say, by
running DFS).  It is well known that a tree of $t$ vertices can be
split into two subtrees each with at most $\floor{2t/3}+2$ vertices
(see, e.g., \cite{???}).  Moreover, if $t \geq 24$, each such subtree
contains at most $\frac{3}{4}t$.  Hence, we can decompose $\calT$
recursively into vertex sets whose sizes is between $\frac{k}{\cmax}$
and $\frac{k}{4\cmax}$, assuming that $\frac{k}{\cmax} \geq 24$.
%
Each such vertex set $\calX$ is connected and contains at most
$\frac{k}{\cmax}$ stars.  Hence $\abs{A(\calX)} \leq k$.  Due to
Lemma~\ref{lemma:r} and since $\abs{V(\calX)} \geq \frac{k}{4\cmax}$,
we have that
\[
\frac{\deg_M(V(\calX))}{\deg_{M^*}(V(\calX))} 
\geq \half + \inv{2\cmax} - \frac{2}{k}
~.
\]
Since $\deg_{M^*}(V(\calT)) = \sum_{\calX} \deg_{M^*}(V(\calX))$ and
$\deg_M(V(\calT)) = \sum_{\calX} \deg_M(V(\calX))$, it follows that
\[
\frac{\deg_M(V(\calT))}{\deg_{M^*}(V(\calT))}
\geq \half + \inv{2\cmax} - \frac{2}{k}
~,
\]
and thus $\abs{M \cap A(\calT)} \geq (\frac{\cmax + 1}{2\cmax}
- \frac{2}{k}) \abs{M^* \cap A(\calT)}$.
\end{proof}


\begin{corollary}
\label{cor:local}
There exists a $(\frac{2\cmax}{\cmax+1} +\eps)$-approximation
algorithm for \carpool, for every $\eps>0$.
\end{corollary}


\iffalse %%%%% Rmeoved

\begin{lemma}
\label{lemma:dec}
Given a parameter $k$, a bounded degree connected graph $G$, with a
maximum degree $d$ can be decomposed into connected components of size
at least $k$, and at most $d(k-1) + 1$.
\end{lemma}

\begin{proof}
We give a constructive proof.  Start constructing a connected
component by adding adjacent vertices one by one to the component and
removing them from the graph.  If at some point, removing a vertex
disconnect the graph, add all the small components (of size less than
$k$) to the constructed component and decompose the large components
(of size at least $k$) recursively.  Note that removing a vertex can
break the graph to at most $d - 1$ additional components.
\end{proof}

\begin{theorem}
\label{theorem:local search}
The local search is ($\frac{1}{2} + \frac{1}{2c}
- \varepsilon$)-approximation algorithm.
\end{theorem}

\begin{proof}
Combine lemma~\ref{lemma:dec}, and lemma~\ref{lemma:r}, and observe
that the maximum degree in $G^{M^*}_M$ is $(C + 1)^C$ to get the
desire result.
\end{proof}

\fi %%%%% End removed


\todo[inline]{Add a proper theorem}

We now show that Corollary~\ref{cor:local} is tight.  Consider the
example in Figure~\ref{fig:local search tight}, the example is for the
special case when $k=11$ and $C=2$ but this example can be generalized
in a straightforward manner.  One can verify that as the graph in the
example growth, the ratio between the optimal solution and the local
search solution approaches $\frac{3}{4}$.

\begin{figure}
\begin{center}
\tikzset{local/.style={
green,
dashed,
}}

\tikzset{opt/.style={
red,
solid,
}}

\begin{tikzpicture}[
	default style, 
	every node/.style={default node, solid}, 
	<-,
	level distance=12mm,
	level/.style={sibling distance=100mm/#1},
	level 1/.style={local},
	level 2/.style={opt},
	level 3/.style={local},
	level 4/.style={opt},
	level 5/.style={local},
	level 6/.style={opt},
]

\node{}
child{ node{}
	child { node{}
		child{ node{}
			child{ node{}
				child{ node{}
					child{ node(1){}}
					child{ node(2){}}
				}
			}
			child{ node{}
				child{ node{}
					child{ node(3){}}
					child{ node(4){}}
				}
			}
		}
	}
	child { node{}
		child{ node{}
			child{ node{}
				child{ node{}
					child{ node(5){}}
					child{ node(6){}}
				}
			}
			child{ node{}
				child{ node{}
					child{ node(7){}}
					child{ node(8){}}
				}
			}
		}
	}
};

\begin{scope}[local]
\draw (1) to[bend right] (5);
\draw (2) to[bend right] (6);
\draw (3) to[bend right] (7);
\draw (4) to[bend right] (8);
\end{scope}

\end{tikzpicture}
\caption{A tight example for Theorem~\ref{theorem:local search}. 
The optimal solution is given by the red solid arcs while the local
search solution is given by the green, dashed arcs.  The solution can
be improved by the local search algorithm only if it removes all the
edges.}
\label{fig:local search tight}
\end{center}
\end{figure}


