
\newpage

\section{Figure}
%\label{sec:figures}


%%%%%%%%%%%%%%%%%%%%%%%%%%%%%%%%%


\section{Omitted Proofs}
\label{sec:omitted}


\newpage





%%%%%%%%%%%%%%%%%%%%%%%%%%%%%%%%%

\section{Tight Analysis}
\label{sec:tight}

We now show that Corollary~\ref{cor:local} is tight.  Consider the
example in Figure~\ref{fig:local search tight}, the example is for the
special case when $k = 11$ and $\cmax = 2$ but this example can be
generalized in a straightforward manner.  One can verify that as the
graph in the example growth, the ratio between the optimal solution
and the local search solution approaches $\frac{3}{4}$.

\begin{figure}[h]
\begin{center}
\scalebox{.9}{
\input{fig-local-tight}
}
\caption{A tight example for Corollary~\ref{cor:local}. 
The optimal solution is given by the red solid arcs while the local
search solution is given by the green, dashed arcs.  The solution can
be improved by the local search algorithm only if it removes all the
edges.}
\label{fig:local search tight}
\end{center}
\end{figure}

