As we mentioned in the introduction, the maximum spanning star forest
can not be approximated within a factor of $\frac{31}{32} + \epsilon$
for any $\epsilon > 0$, unless P$=$NP.  The proof, however, does not
hold for the case when the maximum capacity is constant.  In this
subsection we show that the problem remains APX-hard even in this
case.

Formally, the (unweighted) \textsc{Minimum Dominating Set} problem is
defined as follows.  The input is an undirected graph $G = (V,E)$, and
a feasible solution, or a \emph{dominating set}, is a subset
$D \subseteq V$ that dominates $V$ namely such that
$D \cup \bigcup_{v \in D} N(v) = V$, where $N(v)$ is the neighborhood
of $v$.  The goal is to find a minimum cardinality dominating set.
%
\textsc{Minimum Dominating Set-$b$} is the special case of
\textsc{Minimum Dominating Set} in which
the maximum degree of a vertex in the input graph $G$ is bounded by
$b$.  The problem was shown to be APX-hard, for $b \geq 3$, by
Papadimitriou and Yannakakis~\cite{PapYan88}.

We now consider the unweighted and undirected special case of
the \carpool problem.  In this case, the input consists of an
undirected graph $G$ and a capacity function $c$, and the goal is to
find a carpool matching $M$ that maximizes $\abs{P_M}$.

Given an undirected graph $G$, let $D^*$ be a minimum cardinality
dominating set, and let $M^*$ be an optimal carpool matching with
respect to $G$ and the capacity function: $c(v) = \deg(v)$, for every
$v \in V$.

\begin{observation}
$\abs{P_{M^*}} + \abs{D^*} = \abs{V}$ 
\end{observation}
\begin{proof}
Given a carpool matching $M$, observe that $D_M \cup Z_M$ is a
dominating set.  In the other direction, a dominating set $D$ induces
a carpool matching of size $\abs{V \setminus D}$.
\end{proof}

We use this duality to obtain a hardness result for \carpool.

\begin{theorem}
The \carpool problem is APX-hard, even for undirected and unweighted
instances with maximum capacity bounded by $b$, for $b \geq 3$.
\end{theorem}
\begin{proof}
We prove the theorem by presenting an $L$-reduction from
\textsc{Minimum Dominating Set-$b$}.
(For details on $L$-reductions the reader is referred to~\cite{??}.)

We define a function $f$ from \textsc{Minimum Dominating Set-$b$}
instances to \carpool instances as follows: $f(G) = (G,c)$, where
$c(v) = \deg(v)$, for every $v \in V$.  Next, we define a function $g$
that given a carpool matching computes a dominating set as follows:
$g(M) = V \setminus P_{M}$.  Both $f$ and $g$ can be computed in
polynomial time.

Let $D^*$ be an optimal dominating set with respect to $G$, and let
$M^*$ be an optimal carpool matching with respect to $G$ and $c$.
Since $|D^*| \geq \frac{|V|}{b+1}$, it follows that 
\[
\abs{P_{M^*}} \leq b \abs{D^*}
~,
\]
In addition, if $M$ is a carpool matching, we have that
\[
\abs{D_M \cup Z_M} - \abs{D*}
= (\abs{V} - \abs{P_M}) - \abs{D^*}
= \abs{P_{M^*}} - \abs{P_M}
~.
\]
Hence, there is an $L$-reduction from \textsc{Minimum Dominating
Set-$b$} to unweighted and undirected \carpool with bounded capacity
$b$.
\end{proof}
