
In this subsection we give a local search $(2+\eps)$-approximation
algorithm for \carpool.

We need a few definitions before presenting our algorithm.
%
Let $M$ be a feasible carpool matching.  The weight $w_M(v)$ of a
vertex $v$ with respect to $M$ is the sum of the weights of the arcs
in $M$ that are incident on $v$, namely
\[
w_M(v) \eqdf \sum_{(u, v) \in M} w(u, v) + \sum_{(v, u) \in M} w(v, u)
~.
\]
For a subset of vertices $U \subseteq V$ we define
$w_M(U) \eqdf \sum_{v \in U} w_M(v)$.

We now argue that, with respect to any carpool matching $M$, the total
weight of all the vertices is equal to twice the weight of the matching.

\begin{observation}
\label{lm:val-twice}
$\sum_{v \in V} w_M(v) = 2 \sum_{e \in M} w(e)$.
\end{observation}
\begin{proof}
\(
\sum_{v \in V} w_M(v)
= \sum_{v \in V} \sum_{(u, v) \in M} w(u, v) +
    \sum_{v \in V} \sum_{(v, u) \in M} w(v, u) 
= 2 \sum_{e \in M} w(e)
\).
\end{proof}

Denote by $\delta(u,v)$ the difference between the weight of the arc
and the weight of its source vertex, that is:
\[
\delta_M(u, v) \eqdf w(u, v) - w_M(u)
~.
\]
For a subset $S \subseteq A$ of arcs define
$\delta(S) \eqdf \sum_{(u,v) \in S} \delta(u,v)$.

A subset $S_v$ of arcs entering a vertex $v$, whose size is not
greater than the capacity of $v$, is called an \emph{improvement} to
vertex $v$ if $\delta(S_v)$ is greater than the value of $v$.  More
formally, 

\begin{definition}
A subset $S_v \subseteq A \cap (V \times \{v\})$ is
an \emph{improvement} with respect to a carpool matching $M$, if the
following conditions hold:%
\begin{inparaenum}[(i)]
\item $\abs{S_v} \leq c(v)$, and
\item $\delta_M(S_v) > w_M(v)$.
\end{inparaenum}
Furthermore, if there exists an improvement for a vertex $v$, we say
that vertex $v$ can be \emph{improved}.
\end{definition}

Given an arc $(u,v)$, let $\inc(u,v)$ be the set of arcs that incident
$(u, v)$, namely
\[
\inc(u,v) = (N^-(u) \times \{u\}) \cup (\{v\} \times N^+(v))
~.
\]
If $S$ is a set of arcs, then $\inc(S) \eqdf \bigcup_{(u,v) \in
S} \inc(u,v)$.
%
Figure~\ref{fig:defs} depicts all the above definitions.

\begin{figure}
\centering
\input{fig-val-delta}
\caption{In this example $M$ is the set of the blue, dashed arcs.
In this case $w_M(2) = 7$, $w_M(5) = 2$, and $w_M(6) = 0$.  Also,
$\delta_M(2, 3) = 1$ and $\delta_M(6, 3) = 2$.  The set $\set{(2,3),
(6,3)}$ is an \emph{improvement} to vertex 3 and $\inc(2,3)
= \set{(1,2),(4,2),(3,5),(6,3)}$.}
\label{fig:defs}
\end{figure}

We are now ready to describe our local search algorithm
(Algorithm~\ref{alg:grd}).  The algorithm starts with an empty
carpool matching $M$, and in every iteration it looks for a vertex
that can be improved.  If there exists such a vertex, then the
algorithm removes the arcs that are incident on it from $M$, and adds
the arcs that improve the vertex.  The algorithm terminates when no
vertex can be improved.  Figure~\ref{fig:improvement} depicts an
improvement step.

\begin{algorithm}
\caption{
\label{alg:grd}
Local Search
}
$M \gets \emptyset$ \\
\Repeat{$\text{done}$}{
	$\text{done} \leftarrow \textsc{True}$ \\
	\For{$v \in V$}{
		\If{$\exists$~improvment~$S_v$}{
			$M \leftarrow M \setminus \inc(S_v) \cup S_v$	\\
			$\text{done} \leftarrow \textsc{False}$ \\
		}
	}
}
\end{algorithm}

\begin{figure}
\centering
\input{fig-improvement}
\caption[]{
\label{fig:improvement}
\subref{sub:can improved}
A matching that can be improved
\subref{sub:improved}
The matching after improving vertex 3
}
\end{figure}

We proceed to bound the approximation ratio of the algorithm, assuming
termination.

For a vertex $v$ and a set $S$ of edges entering $v$, let $N^-_S(v)
= \set{u : (u,v) \in S}$ be the set in-neighbors corresponding to $S$.

\begin{lemma}
\label{lm:no improve}
Let $M$ be a matching computed by the local search algorithm.  Let $v$
be a vertex with no improvement, and let $S \subseteq N^-(v)$, such
that $\abs{S} \leq c(v)$, then $w(S) \leq w_M(v) + w_M(N^-_S(v))$.
\end{lemma}
\begin{proof}
If no improvement exists, then
\[
w(S) - w_M(N^-_S(v))
=    \sum_{(u,v) \in S} (w(u,v) - w_M(u))
=    \delta_M(S) 
\leq w_M(v)
~.
\]
\end{proof}

To bound the approximation ratio of the algorithm, we use a charging
scheme argument.

\begin{lemma}
If Algorithm~\ref{alg:grd} terminates, then the computed solution is a
$2$-approximation.
\end{lemma}
\begin{proof}
Let $M$ be the matching produced by the algorithm, and let $M^*$ be an
optimal matching.  We load every vertex $v$ with an amount of money
equal to $w_M(v)$, and then we show that this is enough to pay for
every arc in the optimal matching.  Due to
Observation~\ref{lm:val-twice} the total amount of money that we use
is exactly twice the weight of $M$.

Consider a driver $v \in D_{M^*}$, and let $S = (V \times \{v\}) \cap
M^*$.  By lemma~\ref{lm:no improve} we know that $w(S) \leq w_M(v) +
w_M(N^-_S(v))$, thus we can pay for $S$, using the money on $v$ and on
$N^-_S(v)$.  Clearly, these vertices will not be charged again.
\end{proof}

We show that our analysis is tight using in Figure~\ref{fig:grd
worst}.

\begin{figure}
\centering
\input{fig-2-tight}
\caption[]{
\label{fig:grd worst}
GRD algorithm, worst case example: \\
Consider a path with $2n + 1$ arcs,
and alternating arc weights (2 and 1),
if the GRD algorithm selects all the 1 weighted arcs,
then no further improvement can be done and the value of the matching is $n + 1$,
while the optimal matching has value of $2n$.
}
\end{figure}

It remains to consider the running time of the algorithm.

\begin{theorem}
\label{thm:ls}
Algorithm~\ref{alg:grd} is a $2$-approximation algorithm for \carpool,
if edge weights are integral and polynomially bounded.
\end{theorem}
\begin{proof}
First, observe that determining if a vertex $v$ can be improved can be
done efficiently by considering the incoming arcs to $v$ in a
non-increasing order of their $\delta$s, and only ones with positive
values.  A vertex $v$ can be improved, then, if the $\delta$s of the
first $c(v)$ (or less) arcs sum up to more than $w_M(v)$.
%
It follows that the running time of an iteration of the for-loop is
polynomial.  Since the edge weights are integral and polynomially
bounded, the weight of an optimal carpool matching is polynomially
bounded.  The algorithm runs in polynomial time, because in each
iteration the algorithm improves the weight of the matching by at
least one or otherwise it terminates.
\end{proof}

It remains to consider the general case, where weight are not
polynomially bounded.  In this case one can use scaling and rounding
to ensure a polynomial running time in the cost of a $(1+\eps)$ factor
in the approximation ratio.

\begin{theorem}
There exists a $(2+\eps)$-approximation algorithm for \carpool, for
every $\eps > 0$.
\end{theorem}
\begin{proof}
Let $\eps' = \eps/4$.
%
Define the weight function $w'(e)
= \floor{\frac{w(e)}{w_{\max}} \cdot \frac{m}{\eps'}}$.
%
Let $M^*$ and $M'$ be an optimal carpool matchings with respect to $w$
and $w'$, resp.  We have that
\[
w'(M^*)
=    \sum_{e \in M^*} \floor{\frac{w(e)}{w_{\max}} \cdot \frac{m}{\eps'}}
>    \sum_{e \in M^*} \paren{ \frac{w(e)}{w_{\max}} \cdot \frac{m}{\eps'} } - m
=    \frac{m}{\eps' w_{\max}} \cdot w(M^*) - m
~.
\]
Since $w(M^*) \geq w_{\max}$, we have that 
\[
w'(M^*)
>    \frac{m}{\eps' w_{\max}} \cdot (w(M^*) - \eps' w_{\max})
\geq \frac{m}{\eps' w_{\max}} \cdot (1-\eps') w(M^*)
~.
\]
By Theorem~\ref{thm:ls}, our local search algorithm computes a
$2$-approximate carpool matching $M$ on $(G, c, w')$, and this
matching satisfies $w'(M) \geq w'(M')/2$.  Furthermore, since
$w'(M') \geq w'(M^*)$, it follows that $w'(M) \geq w'(M^*)/2$.
Therefore
\[
w(M)
\geq \frac{\eps' w_{\max}}{m} w'(M) 
\geq \frac{1}{2} \frac{\eps' w_{\max}}{m} w'(M^*)
>    \frac{1-\eps'}{2} w(M^*)
\geq \frac{w(M^*)}{2+\eps}
~.
\]
\end{proof}